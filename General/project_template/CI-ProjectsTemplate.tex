\documentclass[anon]{CI}

% The following packages will be automatically loaded:
% amsmath, amssymb, natbib, graphicx, url, algorithm2e

\title[My CI Project]{My beautiful CI Project}

 % Use \Name{Author Name} to specify the name.
 % If the surname contains spaces, enclose the surname
 % in braces, e.g. \Name{John {Smith Jones}} similarly
 % if the name has a "von" part, e.g \Name{Jane {de Winter}}.
 % If the first letter in the forenames is a diacritic
 % enclose the diacritic in braces, e.g. \Name{{\'E}louise Smith}

 % Two authors with the same address
  % \coltauthor{\Name{Author Name1} \Email{abc@sample.com}\and
  %  \Name{Author Name2} \Email{xyz@sample.com}\\
  %  \addr Address}

 % Three or more authors with the same address:
 % \coltauthor{\Name{Author Name1} \Email{an1@sample.com}\\
 %  \Name{Author Name2} \Email{an2@sample.com}\\
 %  \Name{Author Name3} \Email{an3@sample.com}\\
 %  \addr Address}


 % Authors with different addresses:
 \author{\Name{Author Name1} \Email{abc@myemailaddress.com}\\
 \addr Academic address (optional)
 \AND
 \Name{Author Name2} \Email{xyz@myemailaddress.com}\\
 \addr Academic address (optional)
 }

\begin{document}

\maketitle

\begin{abstract}
This is a great project and therefore it has a concise abstract.
\end{abstract}

\begin{keywords}
List of keywords
\end{keywords}


\section{Problem statement and goals}

This is where the content of your paper starts. Remember:
\begin{itemize}
\item Limit the main text (without bibliography and appendices) to 10 pages.
\item Include, either in the main text or the appendices, enough details to convince the lecturers of the project's merits.
\item You should cite or give credit to all material that is not yours (including pictures, books, web pages, code, ...)
\end{itemize}

\section{Previous work}

This is a very important part, because it puts your work in context.

\section{The CI methods}

Do not repeat well-known theory or formulas. Just mention which methods you use and why you choose them, and provide relevant citations.

\section{Results and Discussion}

The main part of the document.

\section{Strengths and weaknesses}

Be critic with your work ...

\section{Conclusions and future work}

The conclusions are not a mere repetition of the abstract. Basically, you should describe ``what you know now that you did \emph{not} before doing the work''. In addition, mention what would be natural follow-up lines of work.

\section*{References}

\bibliography{yourbibfile}


\appendix

\section{Proof of theoretical results} (if applicable)

\section{Implementation details} (if applicable)


\end{document}
